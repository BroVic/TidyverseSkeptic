
\documentclass[11pt]{article}
\usepackage{graphicx}

\usepackage{times}
\usepackage{listings}
\usepackage{parcolumns}

\setlength{\parindent}{0in}
\setlength{\parskip}{0.1in}

\title{R Design Patterns: Base-R vs.\ Tidyverse}
\author{Norman Matloff \\
      Dept. of Computer Science, University of California, Davis}

\begin{document}

\maketitle

This document enables the reader to see at a glance the difference
between base-R and the Tidyverse in common R design settings.

All examples use R's built-in datasets.

\begin{parcolumns}[rulebetween=true]{2}

\textbf{R binary categorization on a vector:}

\colchunk{

\begin{minipage}{0.40\linewidth}

\begin{lstlisting}
NileHiLow <- 
  ifelse(Nile >= 1000,
    'high','low')
\end{lstlisting}

\end{minipage}

}

\hspace{0.1in}

\colchunk{

\begin{minipage}{0.40\linewidth}

\begin{lstlisting}

Nile %>% as.data.frame %>% 
  mutate(
    HighLow = case_when
    (x < 1000~'low',
     x >= 1000~'high')
  ) %>%
  select(HighLow) %>%
  as.vector -> HighLow


\end{lstlisting}

\end{minipage}

}

\end{parcolumns}

\end{document}

